x`\chapter{Background}
\label{sec:background}
Often in computer graphics, it is necessary to process three-dimensional objects. There are many ways to acquire data of the surface of such geometry and more so many techniques to transform and augment that data to a distinct representation. This often non-trivial task is crucial to further process the object in question in later stages of their respective pipelines. Over the years many representations of acquisition data formats and transformation methods, as well as target data formats, have accumulated.\\
In this chapter, various of these techniques and data formats are examined, of which some of them are used in this work as a vehicle for a novel data transformation routine.\\
Initially, classical approaches are examined in section \ref{classic_approaches} which do not rely on artificial intelligence or machine learning methods.
With the recent advances in deep learning, many new approaches have been developed and thus considered in this work. Hence, the general concept of deep learning and neural networks are then reviewed in section \ref{ml_review}.
Subsequently, these more specific machine learning based methods are reassessed in section \ref{ml_approaches} which rely on distinctive statistical features in their initial data format or the dataset itself, thus allowing for the transformation.
\todo{citations for this stuff?}
\section{Classical approaches}
\label{classic_approaches}
\todo{different name}
In this chapter, classical approaches are examined which do not rely on machine
 learning to transform data of objects from one format to another. 

In general, a three-dimensional object may be specified by its representation of the
 surface. This representation varies from unstructured point cloud data in three-dimensional space
  to a graph-based representation like triangle-/ or quad-meshes and even poly meshes
   with more than four neighbors per node. Surface information may also be described
    with voxelized surface representations.  Often the information on an object's
     surface is only partially defined, where some information may be missing. 
     This can be due to acquisition methods, where parts of the object are partially 
     concealed or recorded from insufficient many sides.



\subsection{Transformation methods}
\todo{maybe not needed. only classic approaches as section?}
Starting from these representations of an object's surface, there are many methods 
of transformation to reach another state of representation. 
Transforming the data not only makes further processing easier
 but rather during their process, more data on the surface is 
 computed based on the given input data.

 This task is non-trivial since, from one representation of an object's surface, many reconstructions in the form of another representation are possible. The amount of data of the object's surface can never be perfect, as the resolution is arbitrary, thus leading to many injective reconstruction methods.

One of the earliest, but still well-known algorithms, presented by
 Lorensen et al. \cite{Lorensen:1987:MCH:37402.37422} reconstructs the surface of an object as a
  triangle mesh, given three-dimensional medical data. This data mostly is given as
   a three-dimensional discrete scalar field, also known as voxels. 
During the reconstruction process, the algorithm iterates through all voxels. For each of these positions, up to eight occupied positions of the voxel-grid can be considered. Furthermore, a reconstruction configuration is defined for every configuration of present/non-present positions in the neighborhood of the voxel. Thus, leading to $2^8 = (256)$ possible configurations, where 15 of them are unique, and the rest are generated by the rotation of the base configurations.  Every configuration reconstructs a number of polygons given one voxel configuration. Finally, all polygons created this way are merged into one mesh.

Similar to the marching cubes algorithm, the work presented by 
Bernardini et al. \cite{817351}, the ball-pivoting algorithm, 
interpolates the surface of an object given the input data.
In contrast, the algorithm processes a collection of unstructured 
three-dimensional points in space with their orientation on the
(ground truth) surface and returns a triangle mesh. Starting from
 a seed triangle, for each edge of not already processed triangles
 , a sphere with set radius $r$ revolves around it. If another point
  of the input point cloud intersects with the sphere, a triangle is 
  created out of the endpoints of the considered edge and the newly 
  found point, thus creating a triangle mesh if every edge of each 
  triangle has been considered. 
  As a baseline, results by the process presented in this work
  will be compared to those of the results by the ball-pivoting algorithm.
\begin{enumerate}
  \item poisson surface reconstruction
  \item Fabians \cite{Groh2017}
  \item instant field aligned meshes
  \item Dennis w\cite{bukenberger2018hierarchical}
\end{enumerate}


\section{Machine Learning review}
\todo{different name for section}
\label{ml_review}

\section{Machine Learning based approaches}
\label{ml_approaches}
\todo{different name for section}
\begin{enumerate}
  \item machine learning good way for inference, probably neural network too, given huge amount of data and finding similarities in data
  \item many approaches for surface reconstruction in classic ml
  \item used for self driving cars. Fast solutions
  \item NN recently started to get nice results
  \item many try to transform given input data to voxel based representation
  \item not many directly from point cloud to meshes
  \item range scanner to meshes
  \item end result not meshes?
\end{enumerate}
Papers to cite:
\begin{enumerate}
  \item Convolutional neural network
  \item Semi-Supervised Classification with Graph Convolutional networks
  \item dense 3d object reconstruction from single depth view
  \item PointNet++
  \item deep marching cubes
  \item pixel2mesh
  \item learning a hierarchical latent-variable model of 3d shapes
  \item FlexConv
  \item unsupervised learning of 3d structure
  \item image2 mesh
  \item Surface reconstruction from unorganized Points
\end{enumerate}
\todo{Find non NN ML papers, from the other prof of ML lecture?}
\todo{Saliency for feature detection}
