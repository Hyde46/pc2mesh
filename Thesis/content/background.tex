\chapter{Background}
\label{sec:background}
Often in computer graphics, it is necessary to process three-dimensional objects. There are many ways to acquire data of the surface of such geometry and more so many techniques to transform and augment that data to a distinct representation. This often non-trivial task is crucial to further process the object in question in later stages of their respective pipelines. Over the years many representations of acquisition data formats and transformation methods, as well as target data formats, have accumulated.\\
In this chapter, various of these techniques and data formats are examined, of which some of them are used in this work as a vehicle for a novel data transformation routine.\\
Initially, classical approaches are examined in section \ref{classic_approaches} which do not rely on artificial intelligence or machine learning methods.
With the recent advances in deep learning, many new approaches have been developed and thus considered in this work. Hence, the general concept of deep learning and neural networks are then reviewed in section \ref{ml_review}.
Subsequently, these more specific machine learning based methods are reassessed in section \ref{ml_approaches} which rely on distinctive statistical features in their initial data format or the dataset itself, thus allowing for the transformation.
\section{Classical approaches}
\label{classic_approaches}
\begin{enumerate}
  \item non trivial task  (etwas mathematischer, genauer werden)
  \item have to define your input and output model mathematically nicely
\end{enumerate}
Papers to cite here
\begin{enumerate}
  \item Marching cubes
  \item Fabians \cite{Groh2017}
  \item instant field aligned meshes
  \item Dennis paper\cite{bukenberger2018hierarchical}
\end{enumerate}


\section{Machine Learning review}
\label{ml_review}

\section{Machine Learning approaches}
\label{ml_approaches}
\begin{enumerate}
  \item machine learning good way for inference, probably neural network too, given huge amount of data and finding similarities in data
  \item many approaches for surface reconstruction in classic ml
  \item used for self driving cars. Fast solutions
  \item NN recently started to get nice results
  \item many try to transform given input data to voxel based representation
  \item not many directly from point cloud to meshes
  \item range scanner to meshes
  \item end result not meshes?
\end{enumerate}
Papers to cite:
\begin{enumerate}
  \item Convolutional neural network
  \item Semi-Supervised Classification with Graph Convolutional networks
  \item dense 3d object reconstruction from single depth view
  \item PointNet++
  \item deep marching cubes
  \item pixel2mesh
  \item learning a hierarchical latent-variable model of 3d shapes
  \item FlexConv
  \item unsupervised learning of 3d structure
  \item image2 mesh
  \item Surface reconstruction from unorganized Points
\end{enumerate}
\todo{Find non NN ML papers, from the other prof of ML lecture?}
\todo{Saliency for feature detection}
