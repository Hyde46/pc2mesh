\chapter{Results}
\label{sec:results}

    An indefinite amount of different meshes can represent the same object.
    Consequently, even more meshes exist, which approximate that same object. 
    Therefore, quantifying the quality of reconstructed meshes is a critical aspect for 
    this application, likewise for similar ones. Only by being able to determine applications
    with higher qualitative results facilitate improvement in the field of mesh 
    reconstruction. 
    \emph{points2mesh} and its described configurations $\mathcal{C}_i$ aim to improve on previous work,
    reconstructing meshes from point cloud data. Its advantages and drawbacks, in contrast to similar efforts, 
    or even reconstruction methods working on other data types, rather than point cloud data, have to be clearly
    defined and evaluated.
    Even though the goal of this work is to reconstruct visually pleasing reconstructions for mesh approximations, 
    considering raw numeric evaluations is nevertheless a crucial component.

    In section \ref{subsec:exp}, the experimental setup is described, which tries to assess
    the quality of reconstructed meshes and compare the defined metrics to other reconstruction methods. 
    Subsequently, the results of the experimental setup are presented in section \ref{subsec:results}, where
    each experiment is accompanied by renderings of some of the reconstructed meshes.


\section{Experimental setup}
\label{subsec:exp}

    Finding a proper way to quantify the approximation of a mesh in comparison to its reconstruction is no straightforward task.
    Many different methods exist, each with its advantages and drawbacks. Depending on the specific application like the size of
    the mesh, the number of vertices and edges, or features to compare, optimal methods of comparisons may differ, if any even
    exist. For the proposed network \emph{points2mesh}, several metrics are chosen, which may lead to conclusions about the quality 
    of the reconstructed mesh. Not only are they applicable to \emph{points2mesh}, but to other reconstruction methods as well. 
    Thus, these metrics can be compared to each other, and an argument may be formed about their reconstruction capacity. This 
    section specifies aspects to evaluate of \emph{points2mesh}'s configurations $\mathcal{C}_i$. Furthermore, utilized metrics 
    for the comparison are detailed in the subsequent subsection. Finally, competing methods for reconstruction are specified,
    to define a frame of reference against which \emph{points2mesh} should be able to hold up or even exceed.


\subsection{Evaluation aspects}
    There are several aspects to evaluate \emph{points2mesh}. Not only are network configurations $\mathcal{C}_i$ a crucial consideration,
    but so is training time, computation time, as well as generalization capabilities and starting meshes $\mathcal{M}_i$ (~Ellipsoid, Torus~). Finally,
    attaining visually pleasing meshes is one such goal too, and thus has to be evaluated as well.

    Therefore, the following aspects are chosen to evaluate.
    \begin{itemize}
        \item \textbf{Configurations} $\mathcal{C}_i$: Each configuration, as described in \ref{configurations}, has its purpose 
                for specific reconstruction capabilities. Evaluating every one of them may lead to conlcusions about objective rankings, and thus
                to a choice of best network configuration for reconstructions with the highest quality.
        \item \textbf{Visual quality}: While being subjective, taking a closer look how good the reconstruction seems to represent the ground truth mesh
        is still relevant.
        \item \textbf{Computation time}: Determining how fast such a reconstruction can be calculated may change which method to use depending on the application.
        \item \textbf{Training time}: Measured in iterations, the convergence time offers valuable insight for networkconfigurations $\mathcal{C}_i$ and how it
        learns from the input data.
        \item \textbf{Generalization capabilities}: Evaluating how well configrations $\mathcal{C}_i$ are capable of generalize the problem of reconstruction is a crucial aspect
        for any neural network. 
        \item \textbf{Changing the basemesh $\mathcal{M}_i$}: \emph{points2mesh} is only able to reconstruct meshes of the same \emph{genus} as the input mesh $\mathcal{M}_i$.
        Thus, testing configurations $\mathcal{C}_i$ with different $\mathcal{M}_i$ of higher \emph{genus} like a toroidal shape may lead to different results and 
        reconstructions with higher \emph{genus}.
    \end{itemize}

    While evaluating configurations, the visual quality and computation time can be compared to other methods, convergence time, generalization and different used basemeshes
    are isolated during evaluation due to their nature and thus are only compared to each other.

\subsection{Evaluation metrics}

    Choosing proper evaluation metrics is ambiguous, but several are often chosen in recent work of mesh reconstruction with favorable information content. 
    While some of them may help to lead to objective conclusions about quality, others are subjective in nature. These metrics 
    are described in the following. 

    \subsubsection*{Meshdistance $d_m$}    
    In 3D vision, a commonly used metric to compare the quality of meshes is done with completeness and accuracy of the
    predicted mesh in comparison to the ground truth mesh. Based on uniformly sampled points on both meshes, completeness and
    accuracy can be defined. While the completeness describes the distance of points of
    ground truth to the predicted mesh, accuracy describes the distance of points from predicted mesh to ground truth.
    Stutz et al. \cite{Stutz2017, Stutz2018CVPR} introduce a parallel $C++$ implementation of \emph{meshdistance} which
    calculates accuracy and completeness, given a ground truth mesh and a prediction. Distances are calculated by a
    point-to-triangle method by Christopher Batty\footnote{https://github.com/christopherbatty/SDFGen last visited: \today} (~See chapter \ref{back:ptt})
    \subsubsection*{Chamfer distance $d_C$}
    Similarly to the \emph{meshdistance} method, to calculate the \emph{chamfer distance}, both meshes have to be uniformly sampled. 
    In contrast, only the distance to each nearest neighbor from ground truth to reconstruction, as well as from reconstruction 
    to ground truth, is considered for the metric.
    While the \emph{chamfer distance} is a more simplified version of the meshdistance, it is still able to provide
    valuable information about the quality of reconstructions.
    %\subsubsection*{MAYBE Earthmovers distance}
    \subsubsection*{Hausdorff distance $d_H$}
    The \emph{hausdorff distance} describes a more sophisticated, asymmetric metric for measuring distances between two similar meshes. 
    Anew, the ground truth mesh is sampled with the same amount of samples as the ground truth mesh possesses. For each sample, its 
    closest point on the predicted mesh is calculated, resulting in minimum, maximum values, as well as mean values for the \emph{hausdorff distance} metric.
    \begin{align}
        d_{H}(\mathcal{M}_{gt},\mathcal{M}_i) = \underset{y\in\mathcal{M}_i}{sup} \underset{y\in\mathcal{M}_{gt}}{inf} d(x,y)
    \end{align}
    With $d(x,y)$ describing a function, calculating the distance from $x$ to $y$.
    \subsubsection*{Visual evidence}    
    Though subective, taking a closer look at, how well a reconstructed mesh seems to represent the ground truth mesh is important too. Relying only on 
    numeric values may lead to reconstructions with low values, but not visually pleasing results. This is due to the complicated nature of the problem itself.
    For a given point cloud, an indefinite amount of different reconstructions are possible. Therefore, inspecting the meshes is necessary.

\subsection{Compared methods}

    Restoring polygonal meshes directly from point cloud data is a new domain of geometry reconstruction, 
    which yet has to be explored deeply. Apart from \emph{points2mesh}, no other methods relying on neural 
    networks and deep learning techniques have yet been proposed for this kind of mode of data transformation.
    For this reason, \emph{points2mesh} has to compete against techniques from traditional approaches like
    \emph{ball-pivoting algorithm} \cite{817351}(~BPA~) and \emph{instant field meshes} \cite{Jakob2015Instant} (~IFM~). Furthermore, by adding
    an intermediate step of voxelization of $\mathcal{PC}$, this work also can be compared against 
    \emph{deep marching cubes} \cite{Liao2018CVPR} (~DeepMC~), which transforms voxelized data into meshes based on deep learning approaches and the classical
    marching cubes algorithm.

\begin{itemize}
    \item \todo{include pixel2mesh if possible}
\end{itemize}

\section{Experiment results}
\label{subsec:results}
\todo{Bei jeder subsubsection auch visual quality mit zeigen}
    The evaluation of \emph{points2mesh} and its configuration $\mathcal{C}_i$ is separated into a numerical and visual evaluation.
    In the following subsection \ref{numericeval}, the previously explained metrics are listed and compiled against each other.
    Likewise, \emph{points2mesh}'s reconstructions are listed to comparable methods with the same metrics.
\subsection{Numerical evaluation}
\label{numericeval}
    The following subsections present the evaluation of the previously defined metrics. Since \emph{points2mesh} 
    is trained on the ModelNet dataset, data samples from this dataset are selected for inference. About 20 percent 
    of the provided point clouds are left out of training and thus used as the test set. Therefore, the network has 
    never seen it before. In that way, its quality can be measured against the other methods and is comparable in the
    first place. The other three methods against which \emph{points2mesh} is compared against, should be able to handle
    point cloud data from ModelNet.
\subsubsection*{Distance metric results}
    The first comparison consists of evaluating all distance metrics $[d_m, d_C, d_H]$ for every configuration
    $\mathcal{C}_i$ and \emph{deep marching cubes}, \emph{ball-pivoting algorithm} and \emph{instant field meshes}.
    One hundred point clouds of each object category are taken at random from the ModelNet dataset. For each method,
    an approximation mesh $M_{p}$ is calculated, and all distance measure metrics to its ground truth mesh computed.
    The mean value of these metrics is noted in the tables. While the first table denotes reconstruction from 1024 
    samples, the following denotes reconstruction from 7500 points.
\begin{center}
    \captionof{table}{Distance metric evaluations with 256 samples per point cloud. Caluclated with \emph{meshdistance} metric. Lower values are better.} \label{tab:distance1024} 
    \begin{tabular}{| l  | c  | c | c | c | c | c |}
        \hline
        %airplane
        Object& $\mathcal{C}_1$ & DMC & IFM & BPA \\ \hline
        \multirow{1}{*}{airplane}&0.093&&0.153&\\\hline
        \multirow{1}{*}{bed}&0.028&&0.123&\\\hline 
        \multirow{1}{*}{bottle}&0.016&&0.134&\\\hline
        \multirow{1}{*}{bowl}&0.044&&0.133&\\\hline
        \multirow{1}{*}{car}&0.035&&0.199&\\\hline
        \multirow{1}{*}{chair}&0.038&&0.188&\\\hline
        \multirow{1}{*}{guitar}&0.027&&0.325&\\\hline
        \multirow{1}{*}{toilet}&0.082&&0.135&\\\hline\hline
        \multirow{1}{*}{mean}&0&c&0&\\        \hline
    \end{tabular}
\end{center}
\begin{center}
    \captionof{table}{Distance metric evaluations with 1024 samples per point cloud. Caluclated with \emph{meshdistance} metric. Lower values are better.} \label{tab:distance1024} 
    \begin{tabular}{| l  | c | c | c | c | c | c | c |}
        \hline
        %airplane
        Object& $\mathcal{C}_1$ & $\mathcal{C}_2$ & $\mathcal{C}_3$ & $\mathcal{C}_4$ & DMC & IFM & BPA \\ \hline
        \multirow{1}{*}{airplane}&0.011&0.091&0.014&0.011&c&0.10&\\\hline
        \multirow{1}{*}{bed}&0.016&0.012&0.018&0.016&c&0.099&\\        \hline
        \multirow{1}{*}{bottle}&0.010&0.007&0.010&0.008&c&0.10&\\        \hline
        \multirow{1}{*}{bowl}&0.025&0.03&0.029&0.022&c&0.097&\\        \hline
        \multirow{1}{*}{car}&0.021&0.017&0.024&0.021&c&0.163&\\        \hline
        \multirow{1}{*}{chair}&0.018&0.012&0.020&0.017&c&0.172&\\        \hline
        \multirow{1}{*}{guitar}&0.014&0.01&0.022&0.017&c&0.234&\\        \hline
        \multirow{1}{*}{toilet}&0.029&0.016&0.028&0.025&c&0.109&\\        \hline\hline
        \multirow{1}{*}{mean}&0.018&0.024&0.021&0.018&c&0.134&\\        \hline
    \end{tabular}
\end{center}
\begin{center}
    \captionof{table}{Distance metric evaluations with 7500 samples per point cloud. Caluclated with \emph{meshdistance} metric. Lower values are better.} \label{tab:distance1024} 
    \begin{tabular}{| l  | c | c | c | c | c | c | c |}
        \hline
        %airplane
        Object& $\mathcal{C}_1$ & $\mathcal{C}_2$ & DMC & IFM & BPA \\ \hline
        \multirow{1}{*}{airplane}&0.015&&&&\\\hline
        \multirow{1}{*}{bed}&0.016&&&&\\\hline
        \multirow{1}{*}{bottle}&0.011&&&&\\\hline
        \multirow{1}{*}{bowl}&0.021&&&&\\\hline
        \multirow{1}{*}{car}&0.019&&&&\\\hline
        \multirow{1}{*}{chair}&0.018&&&&\\\hline
        \multirow{1}{*}{guitar}&0.017&&&&\\\hline
        \multirow{1}{*}{toilet}&0.033&&&&\\\hline
        \multirow{1}{*}{mean}&&&&&\\\hline
    \end{tabular}
\end{center}
\subsubsection*{Generalization capabilities}
    Making use of reconstruction techniques often requires the method to work in a general setting.
    The previous experiments already showed reconstruction from \emph{points2mesh} of test samples,
    which the network has not yet seen. With the same idea of feeding unknown data to the network,
    the next table of evaluated metrics was created. By extending the generalization concept, 
    point cloud samples of objects outside of the trained object categories are fed into the network.
    For that reason, assuming to work on configuration $\mathcal{C}_1$, it has been trained on four different
    object category collections (~See section \ref{sec:categ}~).
    Every trained instance of $\mathcal{C}_1$, now is fed one hundred point cloud samples 
    of \emph{bathtub} and \emph{person}, evaluating every distance metric. Even though, $\mathcal{C}_1$ has never seen any samples of these object categories.

\begin{center}
    \captionof{table}{Distance metric evaluations with 1024 samples per point cloud. Testing with point clouds from outside of trained object categories} \label{tab:distance7500} 
    \begin{tabular}{| l | p{1.1cm} | p{1.2cm} | p{1.2cm} | p{1.2cm} | p{1.2cm} |}
        \hline
        %airplane
        Object category& metric& $\mathcal{C}_1^{big}$ & $\mathcal{C}_1^{small}$ & $\mathcal{C}_1^{airplane}$ & $\mathcal{C}_1^{toilet}$\\ \hline
        \multirow{3}{*}{bathtub}&$d_m$&b&c&c&\\
        &$d_C$&b&c&c&\\
        &$d_H$&b&c&c&\\
        \hline
        \multirow{3}{*}{person}&$d_m$&b&c&c&\\
        &$d_C$&b&c&c&\\
        &$d_H$&b&c&c&\\
        \hline\hline
        %mean vals
        \multirow{3}{*}{mean metrics}&$\bar{d_m}$&b&c&c&\\
        &$\bar{d_C}$&b&c&c&\\
        &$\bar{d_H}$&b&c&c&\\
        \hline
    \end{tabular}
\end{center}

\subsubsection*{Varying basemesh $\mathcal{M}_i$}
    The previously described configurations all are trained on a spheroid as initial mesh $\mathcal{M}_i$. 
    By nature of the neural network and the sphere, objects with \emph{genus} higher than zero cannot be
    reconstructed such that the predicted mesh has the same \emph{genus}. Since it stays constant, 
    changing the initial mesh to a toroidal mesh may lead to possible reconstructions with \emph{genus} one. 
    For the following evaluation, $\mathcal{C}_{1}$ is trained on such an initial toroidal mesh.

\begin{center}
    \captionof{table}{Distance metric evaluations with 1024 samples per point cloud. Initial mesh (~torus mesh~) with a \emph{genus} of one.} \label{tab:distance7500} 
    \begin{tabular}{| l | p{1.1cm} | p{1.2cm} |}
        \hline
        %airplane
        Object category& metric& $\mathcal{C}_1$\\ \hline
        \multirow{3}{*}{chair}&$d_m$&b\\ 
        &$d_C$&b\\
        &$d_H$&b\\
        \hline
    \end{tabular}
\end{center}

\subsubsection*{Noise robustness}

\subsubsection*{Training and evaluation time}
\todo{Den teil hier, falls zeit :)}
For real time applications like autonomous cars, which rely on ...

