\chapter{Methods}
\label{sec:methods}
Two distinct deep-learning network configurations are proposed in this work,
 gaining knowledge from unstructured point cloud data with normal orientation,
  learn a generalized form of them, and infer new shapes from unseen
   points in space as polygonal tri-meshes.

Section \ref{networkconfig} specifies these configurations as well as
 proper objective functions in detail which operate the two-part
  convolutional network.

Subsequentially, section \ref{dataset} specifies the dataset used to
 train the networks, as well as how it was designed and constructed.

\section{Network configuration}
Choosing suitable configurations, proper hyperparameters,
 or even an appropriate dataset for a neural network is not a clear-cut decision. 
 Thus, several configurations and hyperparameter combinations have been developed.
  Subsection \ref{configs} describes these configurations in more detail, followed by a
   closer look of used objective functions in subsections \ref{loss}.
\todo{if misc is still needed, reference it}
\label{networkconfig}
\todo{Kurze zusammenfassunge was passiert im folgenden}
\subsection{My configurations}
\label{configs}
\todo{Full network nn + features. Maybe the simple one too}
\subsection{Loss functions}
\label{loss}
\todo{Objective functions durchgehen mit schaubildern ;)}
\subsection{Misc}
\label{misc}
\todo{subsampling, kd tree, maybe layers durchbringen}
\section{Dataset}
\label{dataset}
\todo{Kurze zusammenfassunge was passiert im folgenden}
\subsection{datageneration}
\subsection{examples}
