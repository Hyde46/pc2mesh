\chapter{Evaluation}
\label{chap:evaluation}

    An indefinite amount of different meshes can represent the same object.
    Consequently, even more meshes exist, which approximate that same object. 
    Therefore, quantifying the quality of reconstructed meshes is a critical aspect for 
    this application, likewise for similar ones. Only by being able to determine applications
    with higher qualitative results facilitate improvement in the field of mesh 
    reconstruction. 
    \emph{points2mesh} and its described configurations $\mathcal{C}_i$ aim to improve on previous work,
    reconstructing meshes from point cloud data. Its advantages and drawbacks, in contrast to similar efforts, 
    or even reconstruction methods working on other data types, rather than point cloud data, have to be clearly
    defined and evaluated.
    Even though the goal of this work is to reconstruct visually pleasing reconstructions for mesh approximations, 
    considering raw numeric evaluations is nevertheless a crucial component.

    In section \ref{subsec:exp}, the experimental setup is described, which tries to assess
    the quality of reconstructed meshes and compare the defined metrics to other reconstruction methods. 
    Subsequently, the results of the experiment are presented in section \ref{subsec:results}, where
    each one is accompanied by tabulated results.


\section{Experimental setup}
\label{subsec:exp}

    Finding a proper way to quantify the approximation of a mesh in comparison to its reconstruction is no straightforward task.
    Many different methods exist, each with its advantages and drawbacks. Depending on the specific application like the size of
    the mesh, the number of vertices and edges, or features to compare, optimal methods of comparisons may differ, if any even
    exist. For the proposed network \emph{points2mesh}, several metrics are chosen, which may lead to conclusions about the quality 
    of the reconstructed mesh. Not only are they applicable to \emph{points2mesh}, but to other reconstruction methods as well. 
    Thus, these metrics can be compared to each other, and an argument may be formed about their reconstruction capacity. This 
    section specifies aspects to evaluate of \emph{points2mesh}'s configurations $\mathcal{C}_i$. Furthermore, utilized metrics 
    for the comparison are detailed in the subsequent subsection. Finally, competing methods for reconstruction are specified,
    to define a frame of reference against which \emph{points2mesh} should be able to hold up or even exceed.
    Finally, a collection of inference examples of the second approach for network training as detailed in \ref{trainings} are listed as well.


\subsection{Evaluation aspects}
    There are several aspects to evaluate \emph{points2mesh}. Not only are network configurations $\mathcal{C}_i$ a crucial consideration,
    but so is computation time, as well as generalization capabilities and starting meshes $\mathcal{M}_i$ (~Ellipsoid, Torus~). 
    Finally,
    attaining visually pleasing meshes is one such goal too, and thus has to be evaluated as well.

    Therefore, the following aspects are chosen to evaluate.
    \begin{itemize}
        \item \textbf{Configurations} $\mathcal{C}_i$: Each configuration, as described in \ref{configurations}, has its purpose 
                for specific reconstruction capabilities. Evaluating every one of them may lead to conlcusions about objective rankings, and thus
                to a choice of best network configuration for reconstructions with the highest quality.
        \item \textbf{Input sample size}: Based on the input sample size of the point cloud, harder, or easier tasks for deforming the ellipsoid are assumed. 
                Evaluating the quality of the reconstructions may lead to conclusions about the amount of necessary samples for proper inference.
                The input sample sizes are $\in[256,1024,7500]$. 
        \item \textbf{Visual quality}: While being subjective, taking a closer look how good the reconstruction seems to represent the ground truth mesh
        is still relevant.
        \item \textbf{Computation time}: Determining how fast such a reconstruction can be calculated may change which method to use depending on the application.
        %\item \textbf{Training time}: Measured in iterations, the convergence time offers valuable insight for networkconfigurations $\mathcal{C}_i$ and how it
        %learns from the input data.
        \item \textbf{Generalization capabilities}: Evaluating how well configrations $\mathcal{C}_i$ are capable of generalize the problem of reconstruction is a crucial aspect
        for any neural network. 
        \item \textbf{Changing the basemesh $\mathcal{M}_i$}: \emph{points2mesh} is only able to reconstruct meshes of the same \emph{genus} as the input mesh $\mathcal{M}_i$.
        Thus, testing configurations $\mathcal{C}_i$ with different $\mathcal{M}_i$ of higher \emph{genus} like a toroidal shape may lead to different results and 
        reconstructions with higher \emph{genus}.
        \item \textbf{Noise robustness}: Acquiring unstructured points in three-dimensional space from real-world object often entails noisy data.
         The perfect acquisition is already nearly impossible due to imperfection in the hardware. Thus, training $\mathcal{N}_{recon}$ with noisy data 
         is an essential part of the evaluation.
        \item \textbf{Computation time}: If such reconstruction methods are fit for time critical application, their computation time is important to evaluate.
    \end{itemize}

    While evaluating configurations, the visual quality and computation time can be compared to other methods, generalization and different used basemeshes
    are isolated during evaluation due to their nature and thus are only compared to each other.

\subsection{Evaluation metrics}

    Choosing proper evaluation metrics is ambiguous, but several are often chosen in recent work of mesh reconstruction with favorable information content. 
    While some of them may help to lead to objective conclusions about quality, others are subjective in nature. These metrics 
    are described in the following. 

    \subsubsection*{Meshdistance $d_m$}    
    In 3D vision, a commonly used metric to compare the quality of meshes is done with completeness of the
    predicted mesh in comparison to the ground truth mesh. Based on uniformly sampled points on both meshes, completeness describes the distance of points of
    ground truth to the predicted mesh. 
    Stutz et al. \cite{Stutz2017, Stutz2018CVPR} introduce a parallel $C++$ implementation of \emph{meshdistance} which
    calculates completeness, given a ground truth mesh and a prediction. Distances are calculated by a
    point-to-triangle method by Christopher Batty\footnote{https://github.com/christopherbatty/SDFGen last visited: \today} (~See chapter \ref{back:ptt})
    %\subsubsection*{Chamfer distance $d_C$}
    %Similarly to the \emph{meshdistance} method, to calculate the \emph{chamfer distance}, both meshes have to be uniformly sampled. 
    %In contrast, only the distance to each nearest neighbor from ground truth to reconstruction, as well as from reconstruction 
    %to ground truth, is considered for the metric.
    %While the \emph{chamfer distance} is a more simplified version of the meshdistance, it is still able to provide
    %valuable information about the quality of reconstructions.
    %\subsubsection*{MAYBE Earthmovers distance}
    %\subsubsection*{Hausdorff distance $d_H$}
    %The \emph{hausdorff distance} describes a more sophisticated, asymmetric metric for measuring distances between two similar meshes. 
    %Anew, the ground truth mesh is sampled with the same amount of samples as the ground truth mesh possesses. For each sample, its 
    %closest point on the predicted mesh is calculated, resulting in minimum, maximum values, as well as mean values for the \emph{hausdorff distance} metric.
    %\begin{align}
    %    d_{H}(\mathcal{M}_{gt},\mathcal{M}_i) = \underset{y\in\mathcal{M}_i}{sup} \underset{y\in\mathcal{M}_{gt}}{inf} d(x,y)
    %\end{align}
    %With $d(x,y)$ describing a function, calculating the distance from $x$ to $y$.
    \subsubsection*{Visual evidence}    
    Though subective, taking a closer look at, how well a reconstructed mesh seems to represent the ground truth mesh is important too. Relying only on 
    numeric values may lead to reconstructions with low values, but not visually pleasing results. This is due to the complicated nature of the problem itself.
    For a given point cloud, an indefinite amount of different reconstructions are possible. Therefore, inspecting the meshes is necessary.


\section{Experiment results}
\label{subsec:results}
    The numerical evaluation of \emph{points2mesh} and its configuration $\mathcal{C}_i$ is denoted in a tabulated form.
    Since \emph{points2mesh} is trained on the ModelNet dataset, data samples from this dataset are selected for inference. About 20 percent 
    of the provided point clouds are left out of training and thus used as the test-set. Therefore, the network has 
    never seen it before. In that way, its quality can be measured against the other methods and is comparable in the
    first place. The other two methods against which \emph{points2mesh} is compared against, should be able to handle
    point cloud data from ModelNet.
    In the following sections, the previously explained metrics are listed and compiled against each other.
    Likewise, \emph{points2mesh}'s reconstructions are listed to comparable methods with the same metrics.
    
\subsection{Distance metric results}
    The first comparison consists of evaluating the distance metric $d_m$ for every configuration
    $\mathcal{C}_i$ and \emph{ball-pivoting algorithm}(~\emph{BPA}~) and \emph{instant field meshes}(~\emph{IFM}~).
    One hundred point clouds of each object category are taken at random from the ModelNet dataset. For each method,
    an approximation mesh $M_{p}$ is calculated, and all distance measure metrics to its ground truth mesh computed.
    The mean value of these metrics is noted in the tables. While the first table denotes reconstruction from 256 
    samples, the following denotes reconstruction from 1024 and subsequently from 7500 points.
    Furthermore, \ref{tab:distance256} also describes the distance metric $d_m$ from the hundred object samples to a not yet deformed sphere\footnote{The sphere object is defined by 
    the implicit surface furmula $x^2+\frac{y^2}{2}+z^2=0$ and created in meshlab with a voxel size of 0.055}.
    This baseline for $d_m$ gives a general understanding of $d_m$. 
    \emph{Deep marching cubes} unfortunately transforms and normalizes their reconstructed mesh in such a way,
    that it cannot be numerically evaluated and thus not compared to \emph{points2mesh} and \emph{IFM} or \emph{BPA}.
    However, a visual comparison is possible and still appealing to conduct, since it is the only other deep learning based
    reconstruction method working (~indirectly\footnote{\emph{DMC} preprocesses the point cloud by transforming it into a voxelized representation, thus not a direct
    tranformation from point cloud data to meshes.}~) on point cloud data, transforming it into a poly-mesh.


    First, as seen in figure \ref{tab:distance256} through \ref{tab:distance7500}, reconstructed meshes already differ depending on the employed
    configuration $\mathcal{C}_i$. For each cell, 100 objects of the \emph{ModelNet} test dataset are used for inference and evaluate
    with $d_m$ against the ground truth. Denoted is the mean value of that distance value. 
    The columns indicate the used configuration or method for the reconstruction. 
    The last column describes the mean value of $d_m$ of a basic spheroidal shape evaluated against 
    all test samples. Comparing this value offers a baseline value, which is an approximate upper bound
    for the quality of the reconstructed meshes. A perfectly reconstructed mesh has a $d_m$ value of $0$.

    The initial goal is to figure out the \emph{best} configuration out of the four proposed ones.
    Reaching that state requires preceding evaluation of $d_m$. Each of them has its merits, but finding the configuration,
    generally capable of the best reconstructions is essential. 
    Taking a look at the mean values of $d_m$ in table \ref{tab:distance1024} for each configuration $\mathcal{C}_i$  shows a small 
    advantage of $\mathcal{C}_1$ and $\mathcal{C}_4$ over the other two configurations. 
    Though small, smaller values are still visible throughout. 
    Exemplary, in figure \ref{fig:1}, the resemblance of the reconstructions are pretty close, though with noticeable differences.
    Taking a closer look at the airplane wings in the first row, $\mathcal{C}_1$, $\mathcal{C}_2$ and $\mathcal{C}_4$ reconstruct
    the small winglets correctly at the end of each wing blade. With configuration $\mathcal{C}_3$, they are not noticeable anymore.
    The advantage can also be seen in the reconstruction of chairs with 1024 samples.  Configuration $\mathcal{C}_1$, $\mathcal{C}_2$
    and $\mathcal{C}_4$ do not have the problem with edges in empty space like configuration $\mathcal{C}_3$, as seen in figure \ref{fig:2}.
    The general low values of $\mathcal{C}_2$, with a higher amount of vertices for the reconstructed object can also be confirmed visually,
    if compared to \ref{fig:1c}, \ref{fig:2c} and \ref{fig:3c}. Though representing the features the best, the general smoothness factor of the object is 
    lower than the other configurations.

    Furthermore, comparing the results of $d_m$ between 256, 1024 and 7500 samples of each configuration, a more significant gap 
    between 256 to 1024 is present, than between 1024 and 7500 samples.
    While the mean value of $d_m$ gets lower using 1024 samples rather than 256, this is generally not true anymore for changing 
    from 1024 samples to 7500.
    Notably, in the case of configuration $\mathcal{C}_1$, its mean value for $d_m$ for all object categories is $0.046$ for 256 samples,
    reducing to $0.018$ at 1024 samples, but staying constant for 7500 samples, also at $0.018$. Additionally, configuration $\mathcal{C}_4$
    has worse results at 7500 samples than at 1024 samples.

    Confirming these numerical results with the visual evidence in figure \ref{fig:7}, 
    while a big jump in visual quality is evident, changing the sample size from 256 to 1024, 
    the change to 7500 samples not so much. As seen in \ref{fig:7d}, the airplane even loses a 
    feature (~fron wheel~) at the last step and thus could explain for higher or equal mean values 
    of $d_m$.

\begin{center}
    \captionof{table}{Distance metric evaluations with 256 samples per point cloud for $\mathcal{N}_{recon}$. Caluclated with \emph{meshdistance} metric. Only considering configurations $\mathcal{C}_1$ and $\mathcal{C}_2$. Lower values are better.} \label{tab:distance256} 
    \begin{tabular}{| l  | c  | c | c | c | c || c |}
        \hline
        %airplane
        Object& $\mathcal{C}_1$ & $\mathcal{C}_2$ & $\mathcal{C}_3$ & BPA & $IFM$  & None\\ \hline
        airplane&0.013&&0.011&0.013&0.153&0.251\\\hline
        bed&0.019&&0.018&0.016&0.123&0.163\\\hline 
        bottle&0.011&&0.011&0.007&0.134&0.201\\\hline
        bowl&0.024&&0.018&0.022&0.133&0.148\\\hline
        car&0.20&&0.019&0.020&0.199&0.155\\\hline
        chair&0.019&&0.018&0.027&0.188&0.195\\\hline
        guitar&0.015&&&0.005&0.325&0.338\\\hline
        toilet&0.043&&&0.016&0.135&0.138\\\hline\hline
        mean&&&&0.016&0.200&0.1986\\        \hline
    \end{tabular}
\end{center}

\begin{center}
    \captionof{table}{Distance metric evaluations with 256 samples per point cloud for $\mathcal{N}_{opt}$. Caluclated with \emph{meshdistance} metric. Only considering configurations $\mathcal{C}_1$ and $\mathcal{C}_2$. Lower values are better.} \label{tab:distance256} 
    \begin{tabular}{| l  | c  | c | c | c || c |}
        \hline
        %airplane
        Object& $\mathcal{C}_1$ & $\mathcal{C}_2$ & BPA & $IFM$  & None\\ \hline
        airplane&0.093&0.12&0.013&0.153&0.251\\\hline
        bed&0.028&0.049&0.016&0.123&0.163\\\hline 
        bottle&0.016&0.014&0.007&0.134&0.201\\\hline
        bowl&0.044&0.089&0.022&0.133&0.148\\\hline
        car&0.035&0.058&0.020&0.199&0.155\\\hline
        chair&0.038&0.045&0.027&0.188&0.195\\\hline
        guitar&0.027&0.039&0.005&0.325&0.338\\\hline
        toilet&0.082&0.075&0.016&0.135&0.138\\\hline\hline
        mean&0.045&0.061&0.016&0.200&0.1986\\        \hline
    \end{tabular}
\end{center}

\begin{center}
    \captionof{table}{Distance metric evaluations with 1024 samples per point cloud. Caluclated with \emph{meshdistance} metric. Lower values are better.} \label{tab:distance1024} 
    \begin{tabular}{| l  | c | c | c | c | c | c || c |}
        \hline
        %airplane
        Object& $\mathcal{C}_1$ & $\mathcal{C}_2$ & $\mathcal{C}_3$ & $\mathcal{C}_4$ & BPA & $IFM$ & None\\ \hline
        airplane&0.011&0.091&0.014&0.011&0.089&0.10 & 0.251\\\hline
        bed&0.016&0.012&0.018&0.016&0.008&0.099 & 0.163\\        \hline
        bottle&0.010&0.007&0.010&0.008&0.003&0.10 & 0.201\\        \hline
        bowl&0.025&0.03&0.029&0.022&0.006&0.097& 0.148\\        \hline
        car&0.021&0.017&0.024&0.021&0.009&0.163& 0.155\\        \hline
        chair&0.018&0.012&0.020&0.017&0.008&0.172& 0.195\\        \hline
        guitar&0.014&0.01&0.022&0.017&0.002&0.234 & 0.338\\        \hline
        toilet&0.029&0.016&0.028&0.025&0.006&0.109 & 0.138\\        \hline\hline
        mean&0.018&0.024&0.021&0.018&0.016&0.134 & 0.1986\\        \hline
    \end{tabular}
\end{center}

    In contrast, the traditional methods, as in \emph{BPA} and \emph{IFM}, get better results; the more samples are provided, both
    numerically as well as visually.
    \emph{IFM} seems to generally struggle to reconstruct visually pleasing results when the number of samples is below 7500.
    Not only that but its numerical values of $d_m$ are sometimes higher than a basic spheroid shape in contrast to its baseline 
    object (~See table \ref{tab:distance256} and \ref{tab:distance1024},  mean value for \emph{car}~). Generally, the metrics have
    higher values for \emph{IFM}, while \emph{BPA} and $\emph{N}_{recon}$ are much closer to each other.

    Not only are they closer to each other, but rather \emph{BPA} produces lower mean values than any configuration of  $\emph{N}_{recon}$.
    Though implicating better results for any case, visually for 256 samples and 1024 samples, $\mathcal{C}_1$ looks in some cases more
    organic, or closer to the ground truth than \emph{BPA} (~Compare figures \ref{fig:7b} with \ref{fig:7c}, and figures \ref{fig:1c} 
    with \ref{fig:1g}~).

\begin{center}
    \captionof{table}{Distance metric evaluations with 7500 samples per point cloud. Caluclated with \emph{meshdistance} metric.Only considering configurations $\mathcal{C}_1$ and $\mathcal{C}_2$. Lower values are better.} \label{tab:distance7500} 
    \begin{tabular}{| l  | c | c | c | c | c | c | c |}
        \hline
        %airplane
        Object& $\mathcal{C}_1$ & $\mathcal{C}_4$ & BPA & $IFM$  \\ \hline
        airplane&0.015&0.092&0.090&0.090\\\hline
        bed&0.016&0.015&0.005&0.089\\\hline
        bottle&0.011&0.009&0.001&0.096\\\hline
        bowl&0.021&0.021&0.003&0.091\\\hline
        car&0.019&0.019&0.003&0.153\\\hline
        chair&0.018&0.016&0.001&0.126\\\hline
        guitar&0.017&0.013&0.001&0.212\\\hline
        toilet&0.033&0.028&0.003&0.101\\\hline\hline
        mean&0.018&0.027&0.013&0.119\\\hline
    \end{tabular}
\end{center}


Finally, inspecting reconstructions from \emph{DMC} demonstrates generally bad results. The low amount of samples used, less than 7500,
seems not to be enough for the deep learning based approach to reconstruct a precise object. At figure \ref{fig:51d}, \emph{DMC} somewhat
reconstruct the general outline of the object, though producing big holes in the side of the object, while also hollowing out the interior
of the car. Additionally, reconstructing a guitar, see figure \ref{fig:3h}, a second, independent \emph{blob} of incoherent triangles is 
produces, unrelated to the ground truth. Lastly, as seen in figure \ref{fig:1i}, reconstructing flat surfaces, like the wings of the airplane
is not possible for \emph{DMC}, resulting in big chunks of missing surface information.

\subsection{Generalization capabilities}
    Making use of reconstruction techniques often requires the method to work in a general setting.
    The previous experiments already showed reconstruction from \emph{points2mesh} of test samples,
    which the network has not yet seen. With the same idea of feeding unknown data to the network,
    the next table of evaluated metrics was created. By extending the generalization concept, 
    point cloud samples of objects outside of the trained object categories are fed into the network.
    For that reason, assuming to work on configuration $\mathcal{C}_1$, it has been trained on four different
    object category collections (~See section \ref{sec:categ}~).
    Every trained instance of $\mathcal{C}_1$, now is fed one hundred point cloud samples 
    of \emph{bathtub} and \emph{person}, evaluating every distance metric. Even though, $\mathcal{C}_1$ has never seen any samples of these object categories.


\begin{center}
    \captionof{table}{Distance metric evaluations with 1024 samples per point cloud. Testing with point clouds from outside of trained object categories} \label{tab:generalize} 
    \begin{center}
        \begin{tabular}{| l  | c | c | c | c || c |}
            \hline
            %airplane
            Object& $\mathcal{C}_1$ & $\mathcal{C}_2$ & $\mathcal{C}_3$ & $\mathcal{C}_4$ & None\\ \hline
            bathtub&0.021&0.015&0.067&0.024 & 0.178\\\hline
            person&0.154&0.137&0.181&0.146 & 0.352\\\hline
        \end{tabular}
    \end{center}
\end{center}

    As seen in table \ref{tab:generalize}, all configurations except for $\mathcal{C}_3$ result in low values for $d_m$ if
    compared to the same metrics of no reconstruction at all. Figures \ref{fig:8} and \ref{fig:9} confirm this contention.
    $\mathcal{C}_2$ keeps the same problem as described above, the general roughness of the surface. $\mathcal{C}_1$ and
    $\mathcal{C}_4$ produce rather similar results, while $\mathcal{C}_3$ fails in every regard to reconstructing anything
    close to the ground truth mesh. 

\subsection{Varying basemesh $\mathcal{M}_i$}
    The previously described configurations all are trained on a spheroid as initial mesh $\mathcal{M}_i$. 
    By nature of the neural network and the sphere, objects with \emph{genus} higher than zero cannot be
    reconstructed such that the predicted mesh has the same \emph{genus}. Since it stays constant, 
    changing the initial mesh to a toroidal mesh may lead to possible reconstructions with \emph{genus} one. 
    For the following evaluation, $\mathcal{C}_{1}$ is trained on such an initial toroidal mesh.

\begin{center}
    \captionof{table}{Excerpt of evaluation table. Distance metric evaluations with 1024 samples per point cloud. Initial mesh (~torus mesh~) with a \emph{genus} of one.} \label{tab:basemesh} 
    \begin{center}
        \begin{tabular}{| l  | c | c |}
            \hline
            %airplane
            Object& $\mathcal{C}_1^{torus}$ & $\mathcal{C}_1^{sphere}$ \\ \hline
            airplane&0.019&0.011\\\hline
            toilet&0.032&0.029\\\hline
        \end{tabular}
    \end{center}
\end{center}
    Though resulting in only marginally higher values of $d_m$ (~See table \ref{tab:basemesh}~), starting with a toroidal base mesh does not solve the problem 
    of reconstructing objects with a \emph{genus} higher than 0. The final prediction of $\mathcal{N}_{recon}$, assuming the 
    new base mesh, does not show any signs of a hole anywhere in the mesh, no matter the input point cloud. 

\subsection{Noise robustness}
    Noise often can be found in point cloud data in real-world scenarios, 
    since recorded data is subject to imperfections in the recording equipment.
    Thus, reconstructing objects, even with noisy data, is an interesting subject 
    for $\mathcal{N}_{recon}$. Table \ref{tab:noise} shows similar to the earlier 
    experiments, reconstruction of different objects categories with point cloud data,
    with added jitter noise (~Noise level of 0.01~).
    While $d_m$ metrics are only slightly higher than without noise, the visual evidence 
    shows a noticeable difference. Figure \ref{fig:81} depicts reconstructed airplanes with
    jitter noise in the input point cloud data. The general reconstruction still achieves 
    the approximate contour of the object, however having more problems with edges dangling 
    in empty space.
\begin{center}
    \captionof{table}{Excerpt of evaluation table. Distance metric evaluations with 1024 samples per point cloud. Comparison of configuration $\mathcal{C}_1$ trained with and without noise (\emph{noise level} of 0.01). Inference with noisy data of the same \emph{noise level}.} \label{tab:noise} 
    \begin{center}
        \begin{tabular}{| l  | c | c |}
            \hline
            %airplane
            Object& $\mathcal{C}_1^{noisy}$ & $\mathcal{C}_1^{clean}$ \\ \hline
            airplane&0.090&0.012\\\hline
            bed&0.017&0.011\\\hline
            bottle&0.017&0.016\\\hline
            bowl&0.029&0.025\\\hline
            car&0.023&0.021\\\hline
            chair&0.018&0.018\\\hline
            guitar&0.015&0.014\\\hline
            toilet&0.041&0.016\\\hline\hline
            mean&0.031&0.018\\\hline
        \end{tabular}
    \end{center}
\end{center}
\subsection{Evaluation time}
\todo{hypericum stats and graphics card used}
For real-time applications like guiding autonomous cars through traffic without collision,
 reconstruction should happen virtually instantaneously for it to be advantageous. 
 The computation time may be critical, and thus has been evaluated, as seen in table
  \ref{tab:time}. 
\emph{IFM} split into three configurations. $\emph{IFM}^{500}$ takes 256 samples as input 
and produces a mesh with 500 vertices (~which is approximately the most vertices for the best
 result at 256 samples~). Similar, $\emph{IFM}^{2400}$ and $\emph{IFM}^{10100}$ are defined 
 for 1024 and 7500 samples.
$\hat{t}$ is a mean value of the measurement for 100 reconstructed objects. 
Generally, \emph{IFM} has the fastest computation times at the same amount of samples when compared to $\mathcal{C}_i$ and \emph{BPA}.
\begin{center}
    \captionof{table}{Inference time for configurations $\mathcal{C}$ and computation time for other techniques} \label{tab:time} 
        \begin{tabular}{| c | c |}
            \hline
            %airplane
            method & $\hat{t}$ in seconds \\\hline
            $\mathcal{C}_{1,3}$ & 0.061 \\\hline
             $\mathcal{C}_2$ & 0.15 \\\hline
             $\mathcal{C}_4$ & 0.043 \\\hline
             $IFM^{500}$ & 0.001 \\\hline
             $IFM^{2400}$ & 0.005 \\\hline
             $IFM^{10100}$ & 0.167 \\\hline
             $BPA$ & 0.062 \\\hline
        \end{tabular}
\end{center}


